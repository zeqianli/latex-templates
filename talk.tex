\documentclass{beamer}
%
% Choose how your presentation looks.
%
% For more themes, color themes and font themes, see:
% http://deic.uab.es/~iblanes/beamer_gallery/index_by_theme.html
%
\mode<presentation>
{
  \usetheme{Darmstadt}      % or try Darmstadt, Madrid, Warsaw, ...
  \usecolortheme{default} % or try albatross, beaver, crane, ...
  \usefonttheme{serif}  % or try serif, structurebold, ...
  \setbeamertemplate{navigation symbols}{}
  \setbeamertemplate{caption}[numbered]
} 

\usepackage[english]{babel}
\usepackage[utf8x]{inputenc}

\usepackage{amsmath,amsthm,amssymb}
\usepackage{braket}
\usepackage{enumerate}
\usepackage{hyperref}
\usepackage{graphicx}
\usepackage{physics}

\newcommand{\N}{\mathbb{N}}
\newcommand{\Z}{\mathbb{Z}}
\newcommand{\comb}[2]{\begin{pmatrix} #1 \\ #2 \end{pmatrix}}
\newcommand{\LHS}{\text{LHS}}
\newcommand{\RHS}{\text{RHS}}
\newcommand{\costh}{\cos\theta}
\newcommand{\sinth}{\sin\theta}
\newcommand{\hf}{\frac{1}{2}}
\newcommand{\hfof}[1]{\frac{#1}{2}}
\newcommand{\qt}{\frac{1}{4}}
\newcommand{\qtof}[1]{\frac{1}{4}}
\newcommand{\sqrtt}{\sqrt{2}}
\newcommand{\varof}[1]{\sigma_{#1}^2}
\newcommand{\kb}{{k_B}}
\newcommand{\degree}{^\circ}
\newcommand{\so}{\Rightarrow\ }
\newcommand{\eto}[1]{e^{#1}}
\newcommand{\sqfr}[2]{\sqrt{\frac{#1}{#2}}}
\newcommand{\parafr}[2]{\left(\frac{#1}{#2}\right)}

\title[short-title]{Title}
\author{(Zeqian) Chris Li}
% \institute{University of Illinois at Urbana-Champaign}
\date{Jan 01, 2000}

\begin{document}

\begin{frame}
  \titlepage
\end{frame}

\begin{frame}{Outline}
 \tableofcontents
\end{frame}



\end{document}




% \begin{frame}{Evolution of $\chi$}
% \begin{columns}
% \column{0.5\textwidth}
% \small {p=0.75}
% \begin{figure}
% \centering
% \includegraphics[scale=0.3]{chi-t-p075}
% \end{figure}

% \column{0.5\textwidth}

% \small{p=0.80}
% \begin{figure}
% \centering
% \includegraphics[scale=0.3]{chi-t-p080}
% \end{figure}
% \end{columns}
% \begin{itemize}
% \item $r=10$; averaged over 20 simulations.
% \end{itemize}
% \end{frame}

% \begin{frame}{Evolution of $\chi$; averaged over 20 simulations}
% \begin{columns}
% \column{0.5\textwidth}
% \small{p=0.83}
% \begin{figure}
% \centering
% \includegraphics[scale=0.3]{chi-t-p083}
% \end{figure}

% \column{0.5\textwidth}
% \small{p=0.90}
% \begin{figure}
% \centering
% \includegraphics[scale=0.3]{chi-t-p090}
% \end{figure}
% \end{columns}
% \begin{itemize}
% \item $r=10$; averaged over 20 simulations.
% \end{itemize}
% \end{frame}



% \section{Introduction}

% \begin{frame}{Introduction}

% \begin{itemize}
%   \item Your introduction goes here!
%   \item Use \texttt{itemize} to organize your main points.
% \end{itemize}

% \vskip 1cm

% \begin{block}{Examples}
% Some examples of commonly used commands and features are included, to help you get started.
% \end{block}

% \end{frame}

% \section{Some \LaTeX{} Examples}

% \subsection{Tables and Figures}

% \begin{frame}{Tables and Figures}

% \begin{itemize}
% \item Use \texttt{tabular} for basic tables --- see Table~\ref{tab:widgets}, for example.
% \item You can upload a figure (JPEG, PNG or PDF) using the files menu. 
% \item To include it in your document, use the \texttt{includegraphics} command (see the comment below in the source code).
% \end{itemize}

% % Commands to include a figure:
% %\begin{figure}
% %\includegraphics[width=\textwidth]{your-figure's-file-name}
% %\caption{\label{fig:your-figure}Caption goes here.}
% %\end{figure}

% \begin{table}
% \centering
% \begin{tabular}{l|r}
% Item & Quantity \\\hline
% Widgets & 42 \\
% Gadgets & 13
% \end{tabular}
% \caption{\label{tab:widgets}An example table.}
% \end{table}

% \end{frame}

% \subsection{Mathematics}

% \begin{frame}{Readable Mathematics}

% Let $X_1, X_2, \ldots, X_n$ be a sequence of independent and identically distributed random variables with $\text{E}[X_i] = \mu$ and $\text{Var}[X_i] = \sigma^2 < \infty$, and let
% $$S_n = \frac{X_1 + X_2 + \cdots + X_n}{n}
%       = \frac{1}{n}\sum_{i}^{n} X_i$$
% denote their mean. Then as $n$ approaches infinity, the random variables $\sqrt{n}(S_n - \mu)$ converge in distribution to a normal $\mathcal{N}(0, \sigma^2)$.

% \end{frame}

% \end{document}
